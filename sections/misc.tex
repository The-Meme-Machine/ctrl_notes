\documentclass[../notes.tex]{subfiles}
\graphicspath{{\subfix{../img/}}}
\begin{document}

\section{Other Important Topics}
\subsection{Frequency Domain vs Time Domain}
Computing the convolution integral is much more difficult in the time domain compared to the frequency domain. Using the frequency domain to analyze a system is means that the Laplace transform can be used, simplifying the convolution step significantly. See (\underline{\ref{sec:transfer_func}})

\subsection{Continuous Time vs Discrete Time}
When the controller is calculated by a computer, the system and feedback cannot be evaluated in continuous time. The system can be thought of as proceeding in time-steps.

\subsubsection{Z Transform}
The Z transform is the corollary to the Laplace Transform, only for discrete time where Laplace is for continuous.

\subsection{Optimization}
\subsubsection{Gradient Descent}

\subsection{Singular Value Decomposition}


\end{document}