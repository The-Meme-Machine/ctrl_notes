\documentclass[../notes.tex]{subfiles}
\graphicspath{{\subfix{../img/}}}
\begin{document}



\end{document}

% \section{Programming Exercise: Importance Sampling}
% Estimating the non-linear system
% \begin{align*}
%     X &= U + Z \\
%     Y &= X + W \\
%     Z \sim \mathcal{N}(0,R), \quad W &\sim \mathcal{N}(0,Q), \quad U \sim \text{Unif}(\{-1, 1\})
% \end{align*}
% To derive the estimator, we must calculate the conditional expectation. However, the posterior distribution is difficult to compute.
% \begin{align*}
%     \Bbb{E}[X|Y=y] &= \int x\cdot P_{X|Y}(x|y)dx \\
%     \text{Apply Bayes' Rule} \quad & \Rightarrow P_{X|Y}(x|y) \\
%     \Bbb{E}[X|Y=y]&= \frac{P_{Y|X}\cdot P_{X}}{P_Y} = \frac{\int x \cdot P_{Y|X}\cdot P_{X}}{\int P_{Y|X}\cdot P_{X}}\\
%     P_X &= \text{Mixture of two Gaussians} = \frac{1}{\sqrt{2 \pi Q}}e^{-\frac{(x+1)^2}{2 R}} + \frac{1}{\sqrt{2 \pi Q}}e^{-\frac{(x-1)^2}{2 R}} \\
%     P_{Y|X} &= \text{Gaussian} \sim \mathcal{N}(X,R) = \frac{1}{\sqrt{2 \pi R}}e^{-\frac{(y-x)^2}{2 R}} \\
%     P_Y &= \text{Marginal distribution. must approximate} \\
%     \text{Thus the estimator is } f^N(y) &= \sum^N_{i=1}\frac{X_i \cdot \frac{1}{\sqrt{2 \pi R}}e^{-\frac{(y-X_i)^2}{2 R}}}{\sum^N_{j=1} \frac{1}{\sqrt{2 \pi R}}e^{-\frac{(y-X_j)^2}{2 R}}} = \sum^N_{i=1}\frac{X_i \cdot e^{-\frac{(y-X_i)^2}{2 R}}}{\sum^N_{j=1} e^{-\frac{(y-X_j)^2}{2 R}}}
% \end{align*}


% \pagebreak
% \section{Programming Exercise: SIR Particle Filter}
% The problem considers a robot with 3 degrees of freedom. The state variables and their dynamics are defined as follows, with $u$ being the (constant and known) positional input, and $w$ being the rotational input. The variable $v(t) \sim \mathcal{N}(0,1)$ has three independent components, and represents the state uncertainty.
% \begin{equation*}
%     X(t) = \begin{bmatrix}
%         x(t)\quad\text{"position in x w.r.t origin"} \\ y(t)\quad\text{"position in y w.r.t. origin"} \\ \theta(t)\quad\text{"orientation w.r.t. origin"}
%     \end{bmatrix}, \quad
%     f(X(t)) = \begin{bmatrix}
%         x(t) + u\cos{(\theta(t))} + \alpha v_1(t) \\
%         y(t) + u\sin{(\theta(t))} + \alpha v_2(t) \\
%         \theta(t) + w + \alpha v_3(t)
%     \end{bmatrix}
% \end{equation*}
% The goal is to estimate the position of the robot using two sensors, placed at $(1, 0)$ and $(-1,0)$ with respect to the origin. The random variable $W(t)\sim\mathcal{N}(0,1)$ has two independent components, and represents the sensor noise. The measurement model is therefore:
% \begin{equation*}
%     h(Y(t)) = \begin{bmatrix}
%         \sqrt{(x(t) - 1)^2 + y(t)^2} + \beta W_1(t) \\
%         \sqrt{(x(t) + 1)^2 + y(t)^2} + \beta W_2(t)
%     \end{bmatrix}
% \end{equation*}
% The scaling variables used were set as:
% \begin{equation*}
%     \alpha = 0.1,\quad \beta = 0.5, \quad u=0.1,\quad \theta=0.05
% \end{equation*}
% \subsection{True Initialization, Low Particle Count}
% The particle filter was initialized with mean $\begin{bmatrix}
%    2 &0 &0 &\frac{\pi}{2} 
% \end{bmatrix}$ and an identity covariance matrix. 100 particles were calculated per time step for this configuration.