\documentclass[../notes.tex]{subfiles}
\graphicspath{{\subfix{../img/}}}
\begin{document}

\section{Modeling \& Simulation}
% \subsection{Physical Systems}
\subsection{Reference Frames}
\subsubsection{Inertial Frame}
\subsubsection{Body Frame}
\subsubsection{North-East-Down Frame}
\subsubsection{Intermediate Frames}
\paragraph{Transforming Between Frames}
\paragraph{Wind Frame}
\paragraph{Stability Frame}
\subsubsection{Coriolis Forces} \label{sec:coriolis}

\subsection{Determining the State Vector}
\subsubsection{Position \& Derivatives}
\subsubsection{Rotation \& Derivatives}
\paragraph{Euler Angles}
\paragraph{Direction Cosine Matrices}
\paragraph{Quaternions}
\paragraph{Quaternion Derivatives}
\begin{equation} \label{eq:quat_derivs}
    \dot{q} = \frac{1}{2}\Omega q, \quad \text{where } \Omega = \begin{bmatrix}
        0 &-\omega_x &-\omega_y &-\omega_z \\
        \omega_x &0 &\omega_z &-\omega_y \\
        \omega_y &-\omega_z &0 &\omega_x \\
        \omega_z &\omega_y &-\omega_x &0 \\
    \end{bmatrix}
\end{equation}
\begin{equation*}
    \text{Recall that } \omega^b_{b/v} = \begin{bmatrix}
        p \\ q \\ r
    \end{bmatrix} = \begin{bmatrix}
        \omega_x \\ \omega_y \\ \omega_z
    \end{bmatrix}
\end{equation*}

\paragraph{Converting Between Representations}
There are different methods based on which representation is being converted from/to. For ease, MATLAB has the built-in functions \verb|eul2dcm|, \verb|dcm2quat|, \verb|quat2eul|. There is also the \verb|angle2quat| function.

\subsubsection{Homogenous Transformation Matrix}
The idea behind the homogenous transformation matrix is to combine rotation and translation.

\subsection{Flat Earth Equations of Motion}

\end{document}