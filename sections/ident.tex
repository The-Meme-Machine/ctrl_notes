\documentclass[../notes.tex]{subfiles}
\graphicspath{{\subfix{../img/}}}
\begin{document}

\section{System Identification \& Analysis}
Before we can understand how to control a system, we must first understand how that system changes in response to input. System identification is the process by which a mathematical model of the dynamical system can be constructed. Designing a controller is infinitely easier with such a model. 
\begin{emphasis}
    The system (excluding the controller) is referred to as the "plant".
\end{emphasis}

\subsection{Linear Regression}
\subsection{Linear, Time-Invariant Systems (LTI)}
\subsection{Linear, Time-Varying Systems (LTV)}
\subsection{State-space Models}
\subsection{Modal Analysis}
Eigenvalues (\underline{\ref{sec:eig}}) come in handy for understanding how a system reacts to input.
\subsubsection{Bounded Input, Bounded Output Stability}

\subsection{Frequency Domain Methods}


\end{document}